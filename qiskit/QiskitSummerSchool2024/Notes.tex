\documentclass{article}
\usepackage[utf8]{inputenc}
\usepackage{enumitem}
\usepackage{dsfont}
\usepackage{listings}
\usepackage{xurl} % Breaks up urls to multiple lines, prevents overfull hbox warning
\usepackage{hyperref}
\usepackage{pythonhighlight}

\lstset{breaklines=true} % Break lines in listings
\hypersetup{
    colorlinks=true,
    linkcolor=blue,
    filecolor=magenta,      
    urlcolor=cyan,
}
\newcommand{\ms}{\medskip}
\newcommand{\bs}{\bigskip}


\newlength{\tindent}
\setlength{\tindent}{\parindent}
\setlength{\parindent}{0pt}
\renewcommand{\indent}{\hspace*{\tindent}}

\usepackage{tabularx}
\newcolumntype{Y}{>{\centering\arraybackslash}X}

\title{Qiskit Summer School 2024 Notes}
\author{Taylor Kimmel}

\begin{document}

\maketitle
\tableofcontents


\section*{First Steps}
\label{s:firstSteps}\addcontentsline{toc}{section}{First Steps}

\textbf{Youtube Playlist of the Summer School:}\\
\url{https://www.youtube.com/playlist?list=PLOFEBzvs-Vvr-GzDWlZpAcDpki5jUqYJu}\ms

\textbf{Get Qiskit Installed:}\\
\url{https://docs.quantum.ibm.com/guides/install-qiskit}\\
\textbf{Get IBM Setup:}\\
\url{https://docs.quantum.ibm.com/guides/setup-channel}\ms

TestSetup.py - Makes sure that you can connect to IBM and get a result

\section*{Optimize for Hardware}
\label{s:optimize}\addcontentsline{toc}{section}{Optimize for Hardware}

%Required - \hyperref[ss:transpile]{transpile}: abstracts the quantum circuit into a circuit that can run on target hardware\\
Required - \textit{transpile}[\nameref{ss:transpile}]: abstracts the quantum circuit into a circuit that can run on target hardware\\
Optional - \textit{verify}[\nameref{ss:verify}] circuit with simulation

\ms

Real quantum device constraints:\\
Basis Gate Set\\
    \indent Only a limited set of gates can be executed directly on the hardware\\
    \indent Other gates must be rewritten in terms of these basis gates\ms

Qubit connectivity\\
    \indent Only certain pairs of qubits can be directly interacted with each other\ms

Errors\\
    \indent Each operation has a chance of error, so circuit optimizations can greatly affect performance\ms

Challenge is running abstract circuit on a specific quantum device.\\
The solution is \textbf{transpilation} - convert abstract circuit into ISA (instruction set architecture) circuit

\subsection*{Transpilation}
\label{ss:transpile}\addcontentsline{toc}{subsection}{Transpilation}

\textbf{Transpilation Terms:}\\
\begin{tabularx}{\textwidth}{ | >{\hsize=0.5\hsize}Y | Y | >{\hsize=0.5\hsize}Y | }
    \hline
    Term & Definition & Orchestra Analogy \\
    \hline
    Pass & A standalone circuit or metadata transformation & An instrument \\
    \hline
    Pass Manager & A list of transpiler passes grouped into a logical unit & An instrument section \\
    \hline
    Staged Pass Manager & A list of pass managers, with each one representing a discrete stage of atranspilation pipeline & The conductor \\
    \hline
\end{tabularx}\bs

Transpile a Circuit with Qiskit SDK:
\begin{enumerate}
\item  Choose which device backend you want to target
\item  Create a preset staged pass manager with your desired optimization level
\item  Run the staged pass manager on the circuit
\end{enumerate}

\textbf{Python Code:}

\begin{python}
service = QiskitRuntimeService()  
backend = service.backend("ibm_brisbane")  
pass_manager = generate_preset_pass_manager(optimization_level=3, backend=backend)  
isa_circuit = pass_manager.run(circuit)
\end{python}

\textbf{Transpiler Stages:}
\begin{enumerate}
    \item Initialization\\
        \indent The circuit is prepared for transpilation, e.g., multi-qubit gates are decomposed into two-qubit gates
    \item Layout\\
        \indent The abstract qubits of the circuit are mapped to the physical qubits on the device
    \item Routing\\
        \indent Swap gates are inserted to enable interactions between qubits that are not physically connected
    \item Translation\\
        \indent The gates of the circuit are translated to the basis gate set of the device
    \item Optimization\\
        \indent The circuit is rewritten to minimize its depth (\# of operations) to decrease the effect of errors
    \item Scheduling\\
        \indent Delay instructions are added to align the circuit with the hardware's timing
\end{enumerate}

\subsection*{Verification}
\label{ss:verify}\addcontentsline{toc}{subsection}{Verification}
\textbf{Simulation Tools}
\begin{itemize}
    \item[] Qiskit SDK reference primitives. Exact simulation, but small circuits only and no noise simulation.
    \item[] Qiskit Runtime local testing. Provides "fake" backends to model each quantum machine.
    \item[] Qiskit Aer. Ecosystem project for simulation, including
        \begin{itemize}
            \item larger circuits
            \item stabilizer circuits
            \item noise models
        \end{itemize}
\end{itemize}

\textbf{Problem}: the runtime cost of simulating quantum circuits scales exponentially with the number of qubits.\\
\indent $\sim 50+$ qubits cannot be simulated\ms

Techniques for large circuits:
\begin{enumerate}
    \item Test smaller versions of circuit
    \item Modify circuit so that it becomes classically simulatable: stabilizer circuit aka Clifford circuit
\end{enumerate}

\section*{Execution}
\label{s:execute}\addcontentsline{toc}{section}{Execution}

\textbf{Primitives} encapsulate the output of a quantum circuit\ms

\textbf{Sampler Primitive:}\\
Output is mapping of bitstrings to counts, e.g., $\{$`$0$'$:12,$ `$1$'$: 9\}$\ms

\textbf{Estimator Primitive:}\\
Output is the \textit{expectation value} of an \textit{observable}, e.g., the net spin of a system.\\
Circuit should not include measurements.\ms

To run a circuit on quantum hardware:
\begin{enumerate}
    \item Initialize the Qiskit Runtime service
    \item Choose a hardware backend
    \item Initialize a Qiskit Runtime primitive with your chosen backend
    \item Invoke the primitive with your circuit
\end{enumerate}\ms

\textbf{Example Sampler Code:}
\begin{python}
import numpy as np
from qiskit.circuit.library import IQP
from qiskit.quantum_info import random_hermitian
from qiskit_ibm_runtime import QiskitRuntimeService, SamplerV2 as Sampler

service = QiskitRuntimeService()

backend = service.least_busy(operational=True, simulator=False, min_num_qubits=127)

n_qubits = 127

mat = np.real(random_hermitian(n_qubits, seed=1234)
circuit = IQP(mat)
circuit.measure_all()

pm = generate_preset_pass_manager(backend=backend, optimization_level=1)
isa_circuit = pm.run(circuit)

sampler = Sampler(backend)
job = sampler.run([isa_circuit])
result = job.result()
\end{python}\ms

The input to the Estimator primitive is a list of \textit{primitive unified blocs (PUBs)}. Each PUB consists of:
\begin{itemize}
    \item A single circuit without measurements
    \item One or more observables
    \item (Optional) One or more parameter values
\end{itemize}

\textbf{Example Estimator Code:}
\begin{python}
import numpy as np
from qiskit.circuit.library import IQP
from qiskit.transpiler.preset_passmanagers import generate_preset_pass_manager
from qiskit.quantum_info import SparsePauliOp, random_hermitian
from qiskit_ibm_runtime import QiskitRuntimeService, EstimatorV2 as Estimator

service = QiskitRuntimeService()
backend = service.least_busy(operational=True, simulator=False, min_num_qubits=127)
estimator = Estimator(backend)

n_qubits = 127

mat = np.real(random_hermitian(n_qubits, seed=1234)
circuit = IQP(mat)
observable = SparsePauliOp("Z" * n_qubits)

pm = generate_preset_pass_manager(backend=backend, optimization_level=1)
isa_circuit = pm.run(circuit)
isa_observable = observable.apply_layout(isa_circuit.layout)

job = estimator.run([isa_circuit, isa_observable])
result = job.result()

print(f" > Expectation value: {result[0].data.evs}")
print(f" > Metadata: {result[0].metadata}")
\end{python}\bs

You can customize Qiskit Runtime behaviour\ms

\textbf{Shots}\\
The number of measurements.\\
More shots reduce statistical error but increase running time.\ms

\textbf{Error Suppression}\\
Use dynamical decoupling to reduce decoherence during execution.\\
Caveat: requires delays between gate executions, so it's not always possible.\ms

\textbf{Error Mitigation}\\
Reduce the effects of device noise after execution.
\begin{itemize}
    \item Twirled Readout Error eXtinction (TREX) measurement twirling
    \item Zero Noise Extrapolation (ZNE)
\end{itemize}
Downside: computational overhead\ms

Execution Modes:
\begin{itemize}
    \item Single job
    \item Batch: multiple concurrent jobs
    \item Session: iterative workload
\end{itemize}\ms

Benefits of batch mode versus a single job with multiple circuits:
\begin{itemize}
    \item Better concurrency of classical processing
    \item Get individual results sooner
\end{itemize}\ms

\textbf{Batch Mode:}
\begin{python}
from qiskit_ibm_runtime import SamplerV2 as Sampler, Batch

max_circuits = 100
all_partitioned_circuits = []
for i in range(0, len(circuits), max_circuits):
    all_partitioned_circuits.append(circuits[i : i + max_circuits])
jobs = []
start_idx = 0

with Batch(backend = backend):
    sampler = Sampler()
        for partitioned_circuits in all_partitioned_circuits:
            job = sampler.run(partitioned_circuits)
            jobs.append(job)
\end{python}\ms

Sessions allow iterative workloads, like VQE\\
Run a job, based on result of the job send a new job\ms

\textbf{Sessions:}
\begin{python}
from qiskit_ibm_runtime import EstimatorV2 as Estimator
from qiskit_ibm_runtime import QiskitRuntimeService, Session

# Initialize Qiskit Runtime service
service = QiskitRuntimeService()
backend = service.backend("ibm_brisbane")

with Session(backend=backend):
    estimator = Estimator()
    # invoke the Estimator as usual
\end{python}\ms

\section*{Postprocessing}
\label{s:postprocess}\addcontentsline{toc}{section}{Postprocessing}

Postprocess technique: visualize results\\
With Sampler output, use \textbf{plot\_histogram()} from \textbf{qiskit.visualizations}\ms

Postprocess technique: post-selection\\
Discard outputs known to be incorrect based on prior knowledge, e.g., if you know outputs must match a certain pattern\ms

Postprocess technique: circuit knitting
\begin{enumerate}
    \item During the optimize for hardware step, decompose the problem into \\smaller circuits, i.e., ``circuit cutting"
    \item Execute smaller circuits
    \item ``Knit" the results into a reconstruction of the original circuit's outcome
\end{enumerate}


\end{document}
